%% LyX 1.4.4 created this file.  For more info, see http://www.lyx.org/.
%% Do not edit unless you really know what you are doing.
\documentclass[english]{article}
\usepackage[T1]{fontenc}
\usepackage[latin1]{inputenc}
\setcounter{secnumdepth}{2}
\setcounter{tocdepth}{2}

\makeatletter
\usepackage{babel}
\makeatother
\begin{document}
\tableofcontents{}


\title{XRCed Manual}


\date{Roman Rolinsky <genericsoma@gmail.com>}

\maketitle

\section*{Introduction}

XRCed is a resource editor for editing XRC resource files used in
wxWidgets or wxPython. XRC is an XML-based format for describing the
layout of the interface. When a XRC file it being parsed in the application
(using wxXmlResource class), objects corresponding to the interface
elements defined in .xrc file are created and can be immediately used.
Please refer to the corresponding documentation depending on your
language for details. XRC files are actually normal text files. Structure
of an XML file is defined by a hierarchy of \emph{nodes,} each node
is delimited by a pair of \emph{tags} in a similar manner to HTML
format. XRC format uses a special \texttt{object} tag for defining
interface element nodes, which can contain a number of \emph{attribute}
nodes with tag names corresponding to the attribute names, e.g. \texttt{<pos>100,100</pos>}.
Object nodes always have XML attribute {}``class'' defined, e.g.
\texttt{<object class='wxFrame'>}.


\section*{Getting started}

XRCed interface contains these components:

\begin{description}
\item [{XML\ tree}] a tree representing the structure of the resource
file.
\item [{Attribute\ panel}] a panel with the information about the currently
selected tree item. Class and object name is shown in the upper part
and a notebook with a number of pages is shown below for editing object's
attributes.
\item [{Pull-down\ menu}] is shown when a used right-clicks inside the
tree control. Is is used for creating new elements and basic operations
such as copy/paste.
\item [{Component\ panel}] is a graphical menu for creating new elements
by clicking on icons or by dragging them on the test window.
\item [{Test\ window}] can be shown for an already existing part of the
resource tree to see the resulting appearance of the interface.
\end{description}
Using XRCed is easy once you understand some basic principles. First
of all, it is using a structure-oriented tree-based editing technique,
versus point-and-click method used in most GUI builders. The tree
is actually almost a direct respresentation of the hierarchy of \emph{object}
nodes in XRC file. More precisely, only the nodes corresponding to
visible interface elements are shown. An example of an object node
which is not an interface element is the sizeritem node. 

XRCed tries hard to make accessible the full information contained
in XRC file, so even the nodes which are not shown in the tree must
be represented. This is done by adding their property pages to the
attribute panel.

Secondly, creating new elements is context-dependent. The new elements
are inserted in the tree relative to the current selection (except
for drag-and-drop mode when the place is determined dynamically).
Sometimes there is an ambiguity on the exact position, because some
elements are \emph{containers} which can have children, but it may
be needed to create the new element as a sibling node instead of a
child. Sibling mode can be forced by holding \textbf{Ctrl} key when
inserting a new element. Similarly, a new element can be inserted
as the first child of a container by holding \textbf{Shift} key (default
is to append it as the last child).
\end{document}
